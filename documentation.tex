\documentclass[11pt, a4paper]{article}

\usepackage[hidelinks]{hyperref}
\setlength\parindent{0pt}

\usepackage[skip=10pt plus1pt, indent=0pt]{parskip}
\usepackage{enumitem}

% Title specification
\title{Matrix Solver - Accompanying Documentation}
\author{Martin Klačer \\ University of Galway}

\begin{document}

% Title Page
\maketitle
\newpage

\tableofcontents
\newpage

% Introduction

\section{Introduction}

This project is a linear algebra adjacent work, created with the intention of furthering my understanding of various linear algebra concepts, mainly including matrices.

The main component of the project is a C++ program, capable of performing a number of matrix operations.
The feature set also allows for arbitrary finite (rational, rather than real) precision, using rational numbers over IEEE-754 floating point representations.

This document briefly describes the history/development process, feature set and usage of the program.

% Development

\section{Development}

The idea for the project was sparked by a linear algebra class at the University of Galway, covering matrices and Gauss-Jordan elimination.
At one point, an online tool for reducing matrices to row echelon forms automatically was mentioned, which interested me.

Mainly, as a Computer Science student, I thought to myself - hold on I can do that!
Which lead to the idea of developing a matrix reduction calculator, which then spiralled to what this project is now.

% Development - Python

\subsection{Python}

The first thing to be decided was the programming language.
I opted to use Python, since it's simple and works well for smaller projects and prototypes.

The first step was to internally define what a Matrix would be.
This ended up being a simple 2D list, meaning a list containing other lists.
Along with this, I made sure to create a Matrix validation function, which ensured a variable given to it was indeed a correctly sized 2D list.

Following this, I wrote a few functions to perform the elementary row operations, to aid the elimination algorithm.

Afterwards, I moved on to developing the actual reduction algorithm itself.
I decided to implement two functions, one to reduce any Matrix to a Row Echelon Form, and the second to reduce a Matrix already in REF to its Reduced Row Echelon Form.

The algorithm to reduce a matrix to REF was fairly simple, inspired by my understanding of the topic, lecture notes and the Row Echelon Form Wikipedia page.
The program goes through all the rows in the matrix, and performs the following steps:

\begin{enumerate}
  \item Find the next row with a non-zero scalar at the current pivot position on the diagonal;
  \item if the row isn't in the correct position, swap it so it's correct, to ensure the pivots go from left to right;
  \item if the non-zero scalar isn't preceded by leading zeros, subtract the previous rows to ensure all leading places are zero;
  \item if the pivot position ends up being different from one, multiply the row by this number's inverse to reduce it to one.
\end{enumerate}

Following these steps until the last row/last pivot point was reached seemed to work well enough for the few matrices I was using for testing purposes.

Once the Matrix was in REF, the function to reduce it to RREF wasn't too difficult to implement.
The Python version works in the following manner:

\begin{enumerate}
  \item Go through each row in the given matrix in REF, and:
  \item for any trailing non-zero scalars, subtract a multiple the row with the appropriate pivot one to reduce the numbers to zero.
\end{enumerate}

This approach was a fairly naive, intuitive attempt to transfer my (inherently imprecise and not exactly algorithmic) ability to solve elimination problems into a computer program.
In retrospect, it had a noticable amount of bugs, a few of which I discovered fairly early on as well.
However, there were other issues with this version that bothered me.

% Development - Scope Creep

\subsection{Scope Creep}

One feature I found lacking fairly quickly was the inherent lack of precision given IEE-defined floating point numbers.
Due to how numbers are stored in a computer memory, any floating point number representation will inherently have rounding errors after a few decimal places.
This is mainly because a floating point system is inherently imprecise for rational numbers with infinitely repeating decimal places (such as $1/3$ being approximately $0.\overline{333}...$).

I then decided that, before fixing the present (but not urgent) errors, it would simply be more fun to try to increase the precision of the program.

Luckily, I found, Python has a built in data type for Rational numbers.
My luck was short-lived, though, as multiplying a Python Rational with a Python IEEE floating point number results in a floating point number, and thus, a loss of precision.

What followed were a few of attempts to exclude any and all floating point numbers from my Python code, by converting them to Rational numbers or removing them altogether.
Those were, needless to say, unsuccessful, as a hidden float somewhere kept disturbing my matrices, and so I had to resort to a different solution.

Clearly, the optimal step to take in this case is to rewrite the program in a language which cares about data types, right?
And thus, I decided to use C++ for the next step of this side project.

Now, C++ doesn't have its own Rational data type, which most certainly suggests that the best course of action is simply to implement my own.
Similarly, though C++ does have built-in dynamic list data types, I, for some reason, decided it would be more fun to make a custom data type for a Matrix.

And so, the project turned from a single-file Python script written in an afternoon into a 10-file C++ project with two custom data types, a handful of utility and test functions and a command-line application using said functions.

% Development - C++

\subsection{C++}

In C++, the first steps were, similarly to Python, to define what a Matrix would be, and in addition, what a Rational number would be.

Using the magic of Object Oriented Programming, I created two classes, one for a rational number, and one for a matrix, of any size, of any data type.

% Rational class description
\subsubsection{Rational class}

The Rational data type is internally represented using three values - the numerator, the denominator and a sign.
Both the numerator and denominator are unsigned 32-bit integers, meaning both can go from $0$ up to a maximum value of $2^{32}-1$.
This means the largest (or smallest) number representable using this data type is $\pm\frac{2^{32}-1}{1}$, and the number with the value closest to zero is $\frac{1}{2^{32}-1}$.

Using the sign bit allows the representation to avoid issues with double negatives - 
if both the numerator and denominator were signed, it would lead to duplicate values in cases where the absolute values of each are equal, and the sign is opposite.

I could have also used a signed numerator and an unsigned denominator, but the implementation with an extra byte for the sign leaves more precision (both unsigned 32-bit),
as well as making it easier to work with the sign in other parts of the code.
The sign variable would, technically, only need a single bit of storage, however, since the smallest data type in C++ takes an entire byte, that's what the sign takes as well.

There are a few interesting parts of the Rational number code I will describe in further detail:
\begin{itemize}
  \item \textbf{simplified form},
  \item \textbf{conversion from floating point numbers},
  \item \textbf{overloaded operators}.
\end{itemize}

The other parts of the code are standard, as one might expect.

\paragraph{Simplified Form:}

To make working with instances of the Rational type easier, one basic functionality I implemented is that any Rational number instance will always try to exist in simplified form.
There is an internal, private member function called \texttt{simplify}, which divides both the numerator and denominator with their greatest common divisor, if one greater than one exists.
Every time any operation is performed which changes either the numerator and denominator, this function is called, and therefore, the Rational is kept in its simplified form.

\paragraph{Conversion From Floating Point Numbers:}

Since the Rational class is a proper implementation, it must also naturally include the ability to construct a Rational instance out of different data types.
This allows users to write simpler code, as any data types convertible to Rational can be implicitly converted, without the need for conversion to be explicitly coded.
For this reason, there are conversion functions from an IEEE float, integer number and a C++ built-in character string (representing a fraction, separated with a '/').

The conversion from a float is a bit more interesting than the other ones, which are very straightforward.
The conversion will inherently be imprecise, so for simplicity's sake, I didn't go through the trouble of implementing any particularly complicated algorithms.
Because of the IEEE floating point standard, according to internet sources, the decimal point precision is about seven significant digits.
Therefore, given a floating point number $f$, the conversion simply sets the numerator to be $10^7*f$, and the denominator to $10^7$.

\paragraph{Overloaded Operators:}

So that the Rational class is easier to use, it takes advantage of one C++ feature called operator overloading.
This basically means that the Rational data type can implement custom behavior for any arithmetic or comparison operator, when done on two Rational instances.

This, in addition to the previously mentioned conversions, allows us to write code such as \texttt{Rational r; r += "5/7";}, which makes using the Rational class in code much easier.


% Matrix class description
\subsubsection{Matrix class}

The Matrix class represents a 2D field of values, of arbitrary size (considering the limits of the computer, of course), of any data type.

The class being able to use nearly any data type is achieved using a C++ feature called generic types.
This allows the Matrix class code to use a template type \texttt{T} rather than any specific type, replacing \texttt{T} with any specific type once a Matrix is instantiated.

For the data contained within the matrix, the class needs to keep track of both the data contained within it and the shape - its rows and columns.
Any two matrices populated with the same numbers but with a different distribution of rows and columns will not be equal, therefore, this has to be accounted for.
The Matrix class stores the number of rows and columns as two separate numeric variables, and the data inside it as a contiguous block of memory.

The Matrix class can be indexed into using the \texttt{.at(column, row)} member function, which uses an internal private member function to calculate the flat index within the data block.
The flat index calculation is the following: $I_f = (I_r * C) + I_c$, given the flat index $I_f$, the row index $I_r$, the column index $I_c$ and the number of columns $C$.

The Matrix class overloads operators as well, but to a more limited extent.
Notably, in terms of comparison, only the 'equals' and 'not equal' operators are considered, since matrices don't have a clearly defined total order.
In terms of arithmetic operations, only addition, subtraction, scalar multiplication and matrix multiplication are considered, as division is also not clearly defined.

The Matrix class itself contains no functionality to perform elementary row operations or matrix reduction, as I found it to be out of scope for the class itself.
For the sake of structure and clarity, this functionality is separated into various additional utility functions.


% Utility functions description
\subsubsection{Utility Functions}

The Matrix Utility Functions are contained in the file \texttt{MatrixUtil.hh}. They're organised into two groups (grouped under C++ namespaces): \textbf{MatrixRowOps}, containing functions to perform Elementary Row Operations, and \textbf{MatrixReduce}, containing functions to reduce a matrix to echelon forms and anything adjacent to that.

These functions are split off into a separate place, since I wouldn't consider them properties of a matrix itself. Any matrix altered by an elementary row operation won't be equal to the original matrix. Even though it will be equivalent in terms of describing a solution set to a system of linear equations, it won't be the same matrix, so I wouldn't consider the operations its property.

The \textbf{MatrixRowOps} part contains functions to subtract, multiply, divide and swap rows. All their implementations are very straightforward, going through on an element-by-element basis. The only necessary functions are addition, multiplitation and swap. Subtraction and division were implemented simply out of convenience, as they're used frequently in Gauss-Jordan elimination.

The \textbf{MatrixReduce} functions cover functionality to validate whether a matrix is in REF or RREF, convert a matrix to REF, from REF to RREF and to RREF (utilising the previous two), as well as a function to calculate the inverse of any invertible matrix.

\paragraph{Reduction to REF:}

The function \texttt{toREF} of \textbf{MatrixReduce} is capable of reducing any given Matrix to its Row Echelon Form.

The algorithm works in the following way:

\begin{enumerate}[label*=\arabic*.]
  \item Iterate through the columns in the Matrix, keeping track of where each subsequent pivot point is, either until the last pivot reached (bottom row), or until all columns processed.
  \item For each Iteration, do:
  \begin{enumerate}[label*=\arabic*.]
    \item Consider the next row expected to contain the current pivot point. Go through the columns of this row up to the current column, and if non-zero, attempt to subtract a previously processed pivot point.
    \item Once this is done, check if the current pivot point is non-zero. If not, go through the rest of the column to find a non-zero pivot point.
    \item If no non-zero pivot point found in the column, this means that the pivot at this column doesn't exist, so the column is skipped and the loop skips to the next iteration.
    \item If a non-zero point exists, mark the row index, and switch it with the current row.
    \item At this point, if there are any non-zero leading numbers in said row, go through them and attempt to reduce by subtracting the previously processed corresponding pivot row.
    \item If the pivot point is not equal to one, divide the row by its value to reduce the pivot to one.
  \end{enumerate}
\end{enumerate}

Once this is done, the Matrix will be in Row Echelon Form.

Based off multiple various tests, this works.
The leading number subtraction being performed beforehand as well addresses the edge case where the point at the pivot is zero, but is lead by non-zero numbers, where subtracting those will leave the pivot non-zero.

\paragraph{Converting REF to RREF:}

The function \texttt{REFtoRREF} of \textbf{MatrixReduce} is capable of reducing a given Matrix in Row Echelon Form to its unique Reduced Row Echelon Form.

The algorithm works in the following way:

\begin{enumerate}
  \item Check that the input Matrix is in REF, if not, end processing.
  \item Go through each column, keeping track of the pivot point, and reduce any non-zero numbers above the pivot by subtracting an appropriate multiple of the pivot row.
\end{enumerate}

This successfully converts any Matrix in REF to its RREF.
Since the input Matrix is in REF already, the algorithm can be fairly simple.

\paragraph{Other Functions:}

The function \texttt{toRREF} of \textbf{MatrixReduce} corresponds in functionality to \texttt{toREF}.
For simplicity, modularity and ease of implementation, this function simply first uses \texttt{toREF} to convert the Matrix to REF, then \texttt{REFtoRREF} to get the RREF.

The function \texttt{invert} of \textbf{MatrixReduce} converts a given input Matrix into its inverse, if such an inverse exists.
The process of obtaining the inverse is, again, using elimination to RREF, with the augmented Matrix of the input on the left and the corresponding square identity on the right.
Converting this to RREF gives us an augmented Matrix with the identity on the left, and the corresponding inverse on the right, if an inverse exists.


% Command-line application description
\subsubsection{Command-line Application}

The compiled program can be used in the command line.
The way to execute it is simply by its name, as it takes no command line arguments.

Once executed, the program runs an interactive text-based terminal, allowing the user to enter various commands.
The overview of the commands is as follows:

\begin{itemize}
  \item ref ..... reduce to REF
  \item rref .... reduce to RREF
  \item forms ... reduce and show both REF and RREF
  \item add ..... add two matrices
  \item sub ..... subtract two matrices
  \item mul ..... multiply two matrices
  \item invert .. get the inverse of a given matrix
  \item test .... run program utility test cases
  \item help .... display this help info
  \item exit .... quit the program
\end{itemize}

Running any command that makes use of matrices allows the user to then enter one or more matrices sequentially.

Any Matrix can be entered as a sequence of lines, where each line is a space-separated row of numbers.
Together, the sequence of rows makes the Matrix.
Once done, the user simply types the word 'done' on a new empty row.

The numbers can be entered in the Rational number string format, i.e. optionally a sign, then the numerator, and then optionally a slash symbol and the denominator.
This can be concisely written down as \texttt{[-]<A>[/<B>]}, where the square brackets represent an optional symbol and the angle brackets a mandatory one.

Most commands make use of matrices, and are fairly straightforward.

The \texttt{test} command executes built-in test cases, which validates that the code works as expected.
This can be very helpful when implementing new features, as it helps prevent undesired functionality regression.

The \texttt{help} command displays a help menu with an overview of the commands, and \texttt{exit} quits the program.


% Program Feature Set

\section{Program Feature Set}

The following is a structured overview of the program feature set, split up by class or file:

\begin{itemize}
  \item \textbf{Rational} (Class)
  \begin{itemize}
    \item Represent a rational number as a fraction, with arbitrary decimal point precision.
    \item Convert a string or floating point number into a rational number/fraction.
    \item Output the represented rational number as a string or floating point number.
    \item Keep the rational number in its simplified form at all times.
    \item Allow simple arithmetic operations to be correctly performed on Rational instances.
  \end{itemize}
  \item \textbf{Matrix} (Class)
  \begin{itemize}
    \item Represent a matrix as a 2D field (rows and columns) of any numeric data type.
    \item Create a Matrix instance from either the shape, shape and initial number or an initialiser list.
    \item Allow resizing, printing and obtaining information about a Matrix instance.
    \item Allow addition and multiplication to be performed on two compatible Matrix instances.
  \end{itemize}
  \item \textbf{MatrixRowOps} (Namespace)
  \begin{itemize}
    \item Persorm the selected elementary row operations on a Matrix instance of any numeric type.
  \end{itemize}
  \item \textbf{MatrixReduce} (Namespace)
  \begin{itemize}
    \item Verify that a Matrix instance is in REF or RREF.
    \item Reduce a Matrix instance to REF, from REF to RREF and to RREF.
    \item Calculate the inverse of an invertible Matrix.
  \end{itemize}
\end{itemize}


% Conclusion

\section{Conclusion}

This project has notably furthered both my understanding of linear algebra and matrices and of C++ programming.
It also surprisingly ended up finished, with all the planned features implemented.
The program will also end up being used as the Linear Algebra class extra-credit project, which is an added benefit.

In passing, I thought of various additional features that the program could use, but that ended up being out of scope.
Some of those may be implemented later, in future versions, if this project gets any updates.

Overall, the project was interesting and fun to make, functions as well as I'd expect, and furthered my knowledge and understanding of various concepts in both maths and programming.

\end{document}
